\PassOptionsToPackage{backref=page}{hyperref}
% %%% === ADDITIONAL PACKAGES
% \usepackage{animate}
\usepackage{subcaption}
\usepackage[labelsep=period, font={color=black}, figurename=Рисунок, tablename=Таблица]{caption}
\usepackage{tikz}
\usepackage{cancel}
\usepackage{booktabs}
\usepackage{stmaryrd}
\usepackage[export]{adjustbox}
\usepackage{algorithmic}
\usepackage{fontawesome5}

\usepackage{amsmath,amsfonts,amssymb,amsthm,mathtools} %Шрифты ы формулах

\graphicspath{{../files/}}

% % \LinesNumbered
\newcommand{\theHalgorithm}{\arabic{algorithm}}
% \newcommand{\theHtable}{\thetable}
\usepackage[ruled,vlined]{algorithm2e}
\renewcommand{\algorithmicrequire}{\textbf{Input:}}
\renewcommand{\algorithmicensure}{\textbf{Output:}}
\newcommand{\vect}[1]{\boldsymbol{\mathbf{#1}}}

% %%% === CONDITIONAL PACKAGE
\usepackage{ifthen}

%%% === TEMPLATE
\usepackage{fontspec}




\def\fontpath{../files/fonts_cu/}

\setmainfont{Inter_24pt-Regular.ttf}[
Path={\fontpath},
BoldFont = Inter-24pt-Bold,
ItalicFont = Inter-24pt-Italic,
BoldItalicFont = Inter-24pt-SemiBoldItalic,
SwashFont = {Inter_24pt-SemiBold}, %Фигурные скобки важны!
Ligatures=TeX
]  %Body text main
%Увеличение шрифта математики в 6/5 раз, чтобы совпадал размер с обычным текстом
%прописани наши стандартные размеры шрифтов

\DeclareMathSizes{20}{24}{18}{13.5}
\DeclareMathSizes{12}{14.4}{10.8}{8.1}
\DeclareMathSizes{11}{13.2}{9.9}{7.425}
\DeclareMathSizes{10}{12}{9}{6.75}
\DeclareMathSizes{8}{9.6}{7.2}{5.4}

\setmonofont{RobotoMono-Light}[
Path={\fontpath},
BoldFont = RobotoMono-Regular,
ItalicFont =  RobotoMono-LightItalic,
BoldItalicFont = RobotoMono-Italic,
Ligatures={}
]

\setbeamerfont{title}{series=\bfseries, size=\fontsize{20}{22}}
\setbeamerfont{frametitle}{series=\bfseries, size=\fontsize{20}{22}}
\setbeamerfont{section title}{series=\bfseries, size=\fontsize{30}{30}}
\setbeamerfont{footline}{size=\fontsize{8}{8}}
\setbeamercolor{title}{fg=black}
\setbeamercolor{frametitle}{fg=black}
\setbeamercolor{section title}{fg=black}
\setbeamercolor{figure name}{fg=tdarkgray}


\setsansfont{Inter_24pt-Regular.ttf}[
Path={\fontpath},
BoldFont = Inter-24pt-Bold,
ItalicFont = Inter-24pt-Italic,
BoldItalicFont = Inter-24pt-SemiBoldItalic,
SwashFont = {Inter_24pt-SemiBold}, %Фигурные скобки важны!
Ligatures=TeX
]  %Body text main

\AtBeginDocument{%
	\let\mathbb\relax
	\DeclareMathAlphabet{\mathbb}{U}{msb}{m}{n}%
}

\setbeamertemplate{bibliography item}{\insertbiblabel}
\setbeamertemplate{itemize items}[circle] % For level-1 itemize
\setbeamertemplate{itemize subitem}[circle] % For level-2 (subitems)
\setbeamertemplate{itemize subsubitem}[circle] % For level-3 (subsubitems)


\definecolor{tdarkgrey} {RGB} {149,149,149}
\definecolor{tdarkgray} {RGB} {149,149,149}
\addtobeamertemplate{footline}{\hfill \hbox{\textcolor{tdarkgray}{\insertframenumber} \hspace{9.5mm}}
	\vskip-0pt
}

\usenavigationsymbolstemplate{}



% %%% === ADDITIONAL COMMANDS
\newcommand*{\Scale}[2][4]{\scalebox{#1}{$#2$}}%
\newcommand{\argmin}{\operatornamewithlimits{argmin}}
\newcommand{\argmax}{\operatornamewithlimits{argmax}}
\newcommand{\la}{\langle}
\newcommand{\ra}{\rangle}

\usepackage{pdfpages}

\setbeamerfont{author}{series=\bfseries, size=\fontsize{20}{22}}
\setbeamerfont{institute}{series=\bfseries, size=\fontsize{12}{12}}


%%% === BACKGROUND IMAGE SETUP
\AtBeginDocument{
	\ifthenelse{\isundefined{\cover}}{} % Обложка
	{
		\includepdf{\cover}
	}
	\ifthenelse{\isundefined{\bgimage}}{ % Титульный лист
		% No background image specified
	}{
		\setbeamertemplate{title page}{
			\begin{tikzpicture}[remember picture,overlay]
				\node[anchor=center, xshift=0pt, yshift=0pt] at (current page.center) {
					\includegraphics[width=1.05\paperwidth, height=1.05\paperheight]{\bgimage}
				};
				\node[fill=white, fill opacity=0.7, text opacity=1, inner sep=10pt, rounded corners=10pt] at (current page.center) {
					\begin{minipage}{0.5\paperwidth}
						\centering
						%\usebeamerfont{title}\inserttitle\par
						\usebeamerfont{author}\insertauthor\par
						\vspace*{0.5cm}
						\usebeamerfont{institute}\insertinstitute\par
						\vspace*{0.5cm}
						\usebeamerfont{institute}
						\href{https://fmin.xyz}{\includegraphics[height=0.27cm]{logo.pdf}}\hspace{0.2cm}  
						\href{https://cu25.fmin.xyz}{\faGem[regular]} \hspace{0.04cm} 
						\href{https://github.com/MerkulovDaniil/cu25}{\faGithub} \hspace{0.07cm} 
						\href{https://t.me/fminxyz}{\faTelegram} \par
					\end{minipage}
				};
			\end{tikzpicture}
		}
	}
	
}


\usebackgroundtemplate% Лого цу в правом верхнем углу -- это на самом деле фон
	{%	
		\ifthenelse{\equal{\insertframenumber}{0}}
		{}
		{
			\begin{tikzpicture}[remember picture, overlay]
			\node[anchor=north east, xshift=-8.6mm, yshift=-6mm] at (current page.north east) {
				\includegraphics[width=7.6mm]{logo_cu_new.pdf}%
			};
			\end{tikzpicture}
		}
	}

\geometry{
	paperwidth=192mm,paperheight=108mm,
	left=9.5mm,right=9.5mm, top=9.5mm, %bottom=9.5mm,
	%showframe
}


%\setbeamercolor{frametitle}{fg=black,bg=yellow} % для дебага горизонтального положения frame title можно закрасить коробку
\usepackage{adjustbox}
\makeatletter
\setbeamertemplate{frametitle}{%
  \ifbeamercolorempty[bg]{frametitle}{}{\nointerlineskip}%
  \@tempdima=\dimexpr\textwidth-31mm% Здесь настраивается размер коробки frame title
  \advance\@tempdima by\beamer@leftmargin%
  \advance\@tempdima by\beamer@rightmargin%
  %\hfill% Это здесь было в исходном определении
  \adjustbox{valign=top,left}{ % , frame. Добавлен для вертикального выравнивания в том числе двухстрочных frame title. Сюда можно написать frame, чтобы увидеть коробку
  \begin{beamercolorbox}[sep=0.3cm,left,wd=\the\@tempdima, leftskip=-4.8mm]{frametitle} % leftskip отвечает за то что frame title прижат влево
    \usebeamerfont{frametitle}%
    \vbox{}\vskip-1ex%
    \if@tempswa\else\csname beamer@fteleft\endcsname\fi%
    \strut\insertframetitle\strut\par%
    {%
      \ifx\insertframesubtitle\@empty%
      \else%
      {\usebeamerfont{framesubtitle}\usebeamercolor[fg]{framesubtitle}\strut\insertframesubtitle\strut\par}%
      \fi
    }%
    \vskip-1ex%
    \if@tempswa\else\vskip-.3cm\fi% set inside beamercolorbox... evil here...
  \end{beamercolorbox}%
}
}
\makeatother

\addtobeamertemplate{footline}{}{\vskip14.2mm}%{\vskip14.6mm} % 9.6mm, чтобы цифра была внизу страницы + 4.6mm отступ
\addtobeamertemplate{frametitle}{\vspace*{-5.0mm}}{\vspace*{-2.5mm}} % Здесь можно менять вертикальное выравнивание заголовков слайдов (первый аргумент) и текста после них (второй аргумент)


\AtBeginSection[]{
\begin{frame}
	\vfill\vspace{1cm}
	\centering
	\begin{beamercolorbox}[sep=8pt,center,shadow=true,rounded=true]{title}
		\usebeamerfont{section title}\insertsectionhead\par%
	\end{beamercolorbox}
	\vfill
\end{frame}
}



\makeatletter
\beamer@centeredfalse % выравнивает текст по верху
\beamer@onlytextwidthtrue % Добавляет в columns опцию onlytextwidth, которая делает полную ширину равной textwidth
\makeatother


\makeatletter
\let\orig@column\column
\let\endorig@column\endcolumn

\renewenvironment{column}[1]{%
	\@tempdima=\dimexpr#1-2.5mm % Уменьшаем размер колонок, чтобы осталось место между ними
	\orig@column{\the\@tempdima}% Вызываем оригинальную column
}{%
	\endorig@column
}
\makeatother

%\usepackage{enumitem}


\setlength{\leftmargini}{11pt}





\setbeamertemplate{itemize item}{\color{black}$\raisebox{1pt}{\scalebox{.6}{$\bullet$}}$} 

\makeatletter

\def\enumerate{%
  \ifnum\@enumdepth>2\relax\@toodeep
  \else%
    \advance\@enumdepth\@ne%
    \edef\@enumctr{enum\romannumeral\the\@enumdepth}%
    \advance\@itemdepth\@ne%
  \fi%
  \beamer@computepref\@enumdepth% sets \beameritemnestingprefix
  \edef\beamer@enumtempl{enumerate \beameritemnestingprefix item}%
  \@ifnextchar[{\beamer@@enum@}{\beamer@enum@}}
\def\beamer@@enum@[{\@ifnextchar<{\beamer@enumdefault[}{\beamer@@@enum@[}}
\def\beamer@enumdefault[#1]{\def\beamer@defaultospec{#1}%
  \@ifnextchar[{\beamer@@@enum@}{\beamer@enum@}}
\def\beamer@@@enum@[#1]{% partly copied from enumerate.sty
  \@enLab{}\let\@enThe\@enQmark
  \@enloop#1\@enum@
  \ifx\@enThe\@enQmark\@warning{The counter will not be printed.%
    ^^J\space\@spaces\@spaces\@spaces The label is: \the\@enLab}\fi
  \def\insertenumlabel{\the\@enLab}
  \def\beamer@enumtempl{enumerate mini template}%
  \expandafter\let\csname the\@enumctr\endcsname\@enThe
  \csname c@\@enumctr\endcsname7
  \beamer@enum@}
\def\beamer@enum@{%
  \beamer@computepref\@itemdepth% sets \beameritemnestingprefix
  \usebeamerfont{itemize/enumerate \beameritemnestingprefix body}%
  \usebeamercolor[fg]{itemize/enumerate \beameritemnestingprefix body}%
  \usebeamertemplate{itemize/enumerate \beameritemnestingprefix body begin}%
  \expandafter
    \list
      {\usebeamertemplate{\beamer@enumtempl}}
      {\usecounter\@enumctr%
        \def\makelabel##1{{\hss\rlap{{%
                \usebeamerfont*{enumerate \beameritemnestingprefix item}%
                \usebeamercolor[fg]{enumerate \beameritemnestingprefix item}\textcolor{black}{##1}}}\hss}}}% Здесь изменено, чтобы цифры выравнивались по левому краю. И цвет.
  \beamer@cramped%
  \raggedright%
  \beamer@firstlineitemizeunskip%
}
\def\endenumerate{\ifhmode\unskip\fi\endlist%
   \usebeamertemplate{itemize/enumerate \beameritemnestingprefix body end}}

\makeatother